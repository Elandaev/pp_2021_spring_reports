\documentclass{report}

 \usepackage[T2A]{fontenc}
 \usepackage[utf8]{luainputenc}
 \usepackage[english, russian]{babel}
 \usepackage[pdftex]{hyperref}
 \usepackage[14pt]{extsizes}
 \usepackage{listings}
 \usepackage{color}
 \usepackage{geometry}
 \usepackage{enumitem}
 \usepackage{multirow}
 \usepackage{graphicx}
 \usepackage{indentfirst}
 \usepackage{amsmath}

 \geometry{a4paper,top=2cm,bottom=3cm,left=2cm,right=1.5cm}
 \setlength{\parskip}{0.5cm}
 \setlist{nolistsep, itemsep=0.3cm,parsep=0pt}

 \lstset{language=C++,
 		basicstyle=\footnotesize,
 		keywordstyle=\color{blue}\ttfamily,
 		stringstyle=\color{red}\ttfamily,
 		commentstyle=\color{green}\ttfamily,
 		morecomment=[l][\color{magenta}]{\#}, 
 		tabsize=4,
 		breaklines=true,
   		breakatwhitespace=true,
   		title=\lstname,       
 }

 \makeatletter
 \renewcommand\@biblabel[1]{#1.\hfil}
 \makeatother

 \begin{document}

 \begin{titlepage}

 \begin{center}
 Министерство науки и высшего образования Российской Федерации
 \end{center}

 \begin{center}
 Федеральное государственное автономное образовательное учреждение высшего образования \\
 Национальный исследовательский Нижегородский государственный университет им. Н.И. Лобачевского
 \end{center}

 \begin{center}
 Институт информационных технологий, математики и механики
 \end{center}

 \vspace{4em}

 \begin{center}
 \textbf{\LargeОтчет по лабораторной работе} \\
 \end{center}
 \begin{center}
 \textbf{\Large«Вычисление многомерных интегралов с использованием многошаговой схемы (метод прямоугольников).»} \\
 \end{center}

 \vspace{4em}

 \newbox{\lbox}
 \savebox{\lbox}{\hbox{text}}
 \newlength{\maxl}
 \setlength{\maxl}{\wd\lbox}
 \hfill\parbox{7cm}{
 \hspace*{5cm}\hspace*{-5cm}\textbf{Выполнил:} \\ студент группы 381808-2 \\ Еландаев П.Е. \\
 \\
 \hspace*{5cm}\hspace*{-5cm}\textbf{Проверил:}\\ доцент кафедры МОСТ, \\ кандидат технических наук \\ Сысоев А. В. \\
 }
 \vspace{\fill}

 \begin{center} Нижний Новгород \\ 2021 \end{center}

 \end{titlepage}

 \setcounter{page}{2}

 % Содержание
 \tableofcontents
 \newpage

 % Введение
 \section*{Введение}
 \addcontentsline{toc}{section}{Введение}
 Геометрический смысл определенного интеграла заключается в площади фигуры, которая образуется в осью координат и криволинейной функцией.
 \par На практике не всегда возможно посчитать определенный интеграл формулой Ньютона-Лейбница. В этом случае можно пожертвовать точностью вычисления и посчитать через геометрический смысл. Один из способов вычисления определенного интеграл является метод прямоугольников.
 \par Существует три вариации вычисления определенного интеграла методом прямоугольников, левых, правых и средних прямоугольников.
 \newpage

 % Постановка задачи
 \section*{Постановка задачи}
 \addcontentsline{toc}{section}{Постановка задачи}
 Целью  данной лабораторной работы требуется реализовать последовательную и параллельную версию вычисления определенного многомерного интеграла методом прямоугольников.
 \par Для реализации параллельной версии необходимо использовать средства OMP, TBB и thread.  Также необходимо провести анализ по времени работы программы. Для проверки корректности работы программы необходимо использовать framework GTest.
 \newpage

 % Описание алгоритма
 \section*{Описание алгоритма}
 \addcontentsline{toc}{section}{Описание алгоритма}
 На вход алгоритма поступают:
 \begin{itemize}
 \item Функция f, от которой необходимо найти определенный интеграл
 \item Пределы интегрирования $[a_1,b_1],...,[a_k, b_k]$
 \item Точность N
 \end{itemize}
 \par Описание алгоритма:
 \begin{itemize}
 \item Для каждого кратного интеграла найдем массив значений аргумента в зависимости от требуемой точности $h_k=a_k+(countpoint*N)$.
 \item Вычислим количество итераций, так как необходимо провести вычисления в каждой точке, а функция является многомерной, тогда количество итераций равно всевозможному набору аргументов(каждый с каждым).
 \item Находим аргументы функции, путем: первый аргумент - это начало интегрирования, а последующие определяет маска с функцией которая сдвигает ее на последующие аргументы.
 \item Найдем сумму по всем значениям функции $sum$.
 \item Вычислим площадь подфункциональной кривой методом прямоугольника, высота это значение функции, а ширина это заданная точность, по условию она равна поэтому ее можно «вынести» $result=sum*pow(N,dimension-1)$

 \end{itemize}

 \newpage

 % Описание схемы распараллеливания
 \section*{Описание схемы распараллеливания}
 \addcontentsline{toc}{section}{Описание схемы распараллеливания}

 Идея распараллеливания алгоритма вычисления многомерного интеграла методом прямоугольников состоит в распараллеливании итераций цикла между потоками. Потоки получат группу итераций, а сумма полученных вычислений является численным ответом.
\par Для достижения данного результата в OpenMP существует директива препроцессора $pragma omp parallel for shedule(static) reduction$, а в TBB $tbb::parallel_reduce$.

 \newpage

 % Описание программной реализации
 \section*{Описание программной реализации}
 \addcontentsline{toc}{section}{Описание программной реализации}\
 Входными параметрами всех реализаций алгоритма являются:
 \begin{itemize}
 \item $std::vector<double> start, end$ - пределы интегрирования
 \item Функция $f$
 \item Точность интегрирования $step$
 \end{itemize}

 \par Для вызова последовательной реализации используется метод 
 \begin{lstlisting}
 RecInt(start, end, f, step)
 \end{lstlisting}
 \par Для вызова OMP реализации используется метод:
 \begin{lstlisting}
 RecIntOMP(start, end, f, step)
 \end{lstlisting}
 \par Для вызова TBB реализации используется метод:
 \begin{lstlisting}
 RecIntTBB(start, end, f, step)
 \end{lstlisting}

 \newpage

 % Результаты экспериментов
 \section*{Результаты экспериментов}
 \addcontentsline{toc}{section}{Результаты экспериментов}
 Вычислительные эксперименты проводились на оборудовании со следующей аппаратной конфигурацией:

 \begin{itemize}
 \item Процессор: AMD Athlon(tm) Dual-Core M320 2.10GHz
 \item Оперативная память: 2 ГБ DDR2;
 \item ОС: Windows 7
 \end{itemize}

 \par Тестирование производилось при $N=0.0003$, на отрезках ${(1;2),(1;1.6),(2,3)}$, $f=sin(x_1)*cos(x_2)*x_3$
 \\

 \begin{table}[!h]
 \begin{center}
 \begin{tabular}{lllllll}
 Реализация & Время & Ускорение   \\
 SEQ        & 4.77 & 1           \\
 OMP        & 3.13 & 1.52       \\
 TBB        & 2.72 & 1.75       \\

 \end{tabular}
 \end{center}
 \caption{Результаты вычислительных экспериментов}
 \centering
 \end{table}

 \par Из полученных результатов видно, что удалось получить ускорение в 1.6 раз. Исходя из этого можно сделать вывод, алгоритм можно распараллелить.
 \newpage

 % Заключение
 \section*{Заключение}
 \addcontentsline{toc}{section}{Заключение}
 В результате лабораторной работы были разработаны последовательная и параллельная реализации вычисления определенного многомерного интеграла методом левых прямоугольников.
 \par Были разработанна последовательная версия алгоритма, и на ее основе были полученны параллельные реализации с использованием OpenMP и TBB. 
 \par Из результатов вычислительных экспериментов видно, что параллельные реализации алгоритма являются эффективными и, в среднем обеспечивают ускорение в количество физических ядер процессора.
 \newpage

 % Литература
 \begin{thebibliography}{1}
 \addcontentsline{toc}{section}{Литература}
 \bibitem{Gergel} Гергель В.П. Параллельное программирование с использованием
 OpenMP, 2007, 33 с.
 \bibitem{} «Параллельное программирование на OpenMP»
 \\URL:\url {http://ccfit.nsu.ru/arom/data/openmp.pdf}
 \bibitem{Sidnev} Сиднев А.А., Мееров И.Б., Сысоев А.В. «Разработка параллельных программ в системах с общей памятью с использованием библиотеки Intel Threading Building Blocks (TBB)».
 \\URL:\url {http://hpc-education.ru/files/lectures/meerov/ppt06.pdf}
 \end{thebibliography}
 \newpage

 % Приложение
 \section*{Приложение}
 \addcontentsline{toc}{section}{Приложение}
 \par Последовательная реализация
 \begin{lstlisting}
 void iterplus(std::vector<int> *B, int it, const std::vector<std::vector<double>> &p) {
    int size = p[it].size();
    if (B->operator[](it) == (size - 1)) {
        if ((it-1) < 0)
            return;
        B->operator[](it) = 0;
        iterplus(B, --it, p);
    } else {
        B->operator[](it)++;
    }
}
double RecInt(std::vector<double> start,
              std::vector<double> end,
              std::function<double(std::vector<double>)> f,
              double step) {
    std::vector<int> countstep;
    int size = start.size();
    for (int i = 0; i < size; i++) {
        if (end[i] < start[i])
            throw "wrong segment";
        if (step > end[i] - start[i])
            throw "step is biggest";
        countstep.push_back(static_cast<int>((end[i] - start[i]) / step));
    }
    if (step <= 0)
        throw "wrong splitting: smallest";
    std::vector<std::vector<double> > point;
    point.resize(start.size());
    int countTrial = 1;
    for (int i = 0; i < size; i++) {
        for (int j = 0; j < countstep[i]; j++) {
            point[i].push_back(start[i] + j * step);
        }
        countTrial *= point[i].size();
    }
    double sum = 0.0;
    std::vector<int> B(start.size());
    for (int i = 0; i < size; i++) {
        B[i] = 0;
    }
    std::vector<double> Trial(start.size());
    int dim = static_cast<int>(start.size() - 1);
    for (int i = 0; i < countTrial; i++) {
        for (int j = 0; j < size; j++) {
            Trial[j] = point[j][B[j]];
        }
        iterplus(&B, dim, point);
        sum += f(Trial);
    }
    return sum * (std::pow(step, dim+1));
}

 \end{lstlisting}
 \par Реализация OMP
 \begin{lstlisting}
  double RecIntOmp(std::vector<double> start,
              std::vector<double> end,
              std::function<double(std::vector<double>)> f,
              double step) {
    std::vector<int> countstep;
    int size = start.size();
    for (int i = 0; i < size; i++) {
        if (end[i] < start[i])
            throw "wrong segment";
        if (step > end[i] - start[i])
            throw "step is biggest";
        countstep.push_back(static_cast<int>((end[i] - start[i]) / step));
    }
    if (step <= 0)
        throw "wrong splitting: smallest";
    std::vector<std::vector<double> > point;
    point.resize(start.size());
    int countTrial = 1;
    for (int i = 0; i < size; i++) {
        for (int j = 0; j < countstep[i]; j++) {
            point[i].push_back(start[i] + j * step);
        }
        countTrial *= point[i].size();
    }
    double sum = 0.0;
    std::vector<int> B(size);
    for (int i = 0; i < size; i++) {
        B[i] = 0;
    }
    std::vector<double> Trial(size);
    int dim = size - 1;
    int flag = 1;
#pragma omp parallel for schedule(static) reduction(+:sum) firstprivate(Trial, B, flag) shared(point, dim)
    for (int i = 0; i < countTrial; i++) {
        if (flag == 1) {
            int k = i;
            int count = 0;
            while (k > 0) {
                B[dim-count] = k % point[dim - count].size();
                k = k / point[dim - count].size();
                count++;
            }
            flag = 0;
        }
        for (int j = 0; j < size; j++) {
            Trial[j] = point[j][B[j]];
        }
        iterplus(&B, dim, point);
        sum += f(Trial);
    }
    return sum * (std::pow(step, size));
}
 \end{lstlisting}
 \par Реализация TBB
 \begin{lstlisting}
 double RecIntTbb(std::vector<double> start,
              std::vector<double> end,
              std::function<double(std::vector<double>)> f,
              int countstep) {
    if (countstep < 1)
        throw "wrong splitting: smallest";
    int size = start.size();
    for (int i = 0; i < size; i++) {
        if (end[i] < start[i])
            throw "wrong segment";
    }
    std::vector<double> step(size);
    step.clear();
    for (int i = 0; i < size; i++) {
        step.push_back((end[i] - start[i]) / static_cast<double>(countstep));
    }
    int CountTrial = pow(countstep, size);
    double sum = 0.0;
    std::vector<double> Trial(size);
    sum = tbb::parallel_reduce(
        tbb::blocked_range<int> {0, CountTrial}, 0.f,
        [&](const tbb::blocked_range<int>& r, double loc_res)->double {
            int begin = r.begin();
            int end = r.end();
            for (int i = begin; i < end; i++) {
                for (int j = 0; j < size; j++) {
                    Trial[j] = step[j] * (i % countstep) + start[j];
                }
                loc_res += f(Trial);
            }
            return loc_res;
    }, std::plus<double>() );
    double high = 1.0;
    for (int i = 0; i < size; i++) {
        high *= step[i];
    }
    return sum * high;
}
 \end{lstlisting}
 \end{document}